\chapter*{Introduzione}
\addcontentsline{toc}{chapter}{Introduzione}

Queste note, che seguono l'ottimo testo \cite{brealey_et_al1999}, non contengono nulla di originale,
se non forse il metodo di esposizione.\newline 
Esse nascono dalla volontà di divulgare il calcolo del valore attuale netto come metodo
per la stima degli investimenti, con un approccio più ``orientato alla matematica''
di quello che si fà di solito. Per realizzare questo obiettivo presento la materia in modo assiomatico
nell'accezione di Hermann Weyl: propongo una struttura logica ad assiomi che ha lo scopo di ordinare e
chiarire nozioni e risultati comunque già noti in letteratura. Conseguenza di questa scelta è una
trattazione che è al contempo più astratta ma anche più breve e diretta rispetto a quella offerta
dagli Autori nel testo di riferimento.\newline
Preciso infine che l'approccio assiomatico che ho usato è quello detto ``intuitivo'': il problema della 
coerenza e indipendenza degli assiomi proposti non è neanche lontanamente accennato come non è neppure
offerta una qualsivoglia discussione elementare di quelli che sono i problemi di logica che l'uso
di un sistema assiomatico di questo tipo comporta. Si tratta solo di appunti che vogliono
essere chiari, facili e comodi da consultare.