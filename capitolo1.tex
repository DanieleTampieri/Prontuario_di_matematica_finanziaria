\chapter{Leggi fondamentali}
\label{chap:Leggi_fondamentali}

La definizione seguente, anche se piuttosto informale, è centrale a tutta la trattazione.
\begin{defn}\label{def:cap}
  Il ``\emph{capitale}'' è una risorsa economica, un ``\emph{flusso di cassa}'', un ``\emph{bene}'',
  qualcosa che per tutti ha un ``\emph{valore}'' e che quindi si può misurare, comprare e scambiare.
\end{defn}
I termini indicati nella definizione~\ref{def:cap} verranno da ora in avanti usati come sinonimi.
Chiameremo \emph{denaro} la generica unità di misura del capitale, valore o che dir si voglia.

Ciò premesso, possiamo enunciare il ``principio zero'' dell'economia. 

\begin{legge}\label{law:0}
  Esiste almeno un mercato dei capitali.
\end{legge}
La legge~\ref{law:0} non ha scopo ``operativo'' perché non dice come si realizza questo mercato: dice
semplicemente che esiste un luogo e/o un modo e/o un metodo che permette scambi di capitale, beni, valori.
È anche ovvio che possono esistere, e di conseguenza esistono, più mercati distinti tra loro. Scopo di questa
legge è quindi solo garantire la coerenza logica della teoria che si sviluppa (sic!): si ammette che ciò di cui
si parla esista effettivamente nei termini in cui se ne parla.

\begin{defn}\label{def:inv}
  L'acquisto di un bene disponibile solo ad un tempo futuro rispetto al momento in cui si conlude
  l'acquisto stesso si definisce \emph{investimento}.
\end{defn}
L'oggetto della definizione~\ref{def:inv} è uno dei concetti centrali di queste note. I (semplici) metodi
matematici descritti qui e nel seguito sono stati sviluppati per valutare il valore degli investimenti e quindi
decidere se abbia ragionevolmente senso fare o no un dato investimento.\newline
La locuzione ``ragionevolmente'' significa ``con rischio accettabile di perdere il valore investito'' e il rischio
è definito e quantificato (almeno qualitativamente) dalle tre leggi che sono enunciate qui di seguito. Rimandiamo
quindi la discussione sul suo significato a dopo la loro introduzione: per ora è sufficiente la comprensione intuitiva
che si ha di questo concetto.\newline
Come per la definizione~\ref{def:cap}, i termini che identificano il prezzo dell'investimento (definizione~\ref{def:inv})
verranno da ora in avanti usati come sinonimi.

La prossima legge ha natura diversa rispetto al ``principio zero'' perché stabilisce una relazione tra i valori di capitali
disponibili a tempi diversi, e quindi in definitiva tratta di investimenti (si veda~\cite[§2.1.1 p. 12]{brealey_et_al1999}).

\begin{legge}\label{legge:1}
  Un denaro oggi vale più di un denaro domani.
\end{legge}
La legge \ref{legge:1} formalizza il concetto intuitivo di \emph{rischio di investimento}: il bene di cui si dispone
al presente ha maggior valore rispetto a uno di ugual valore nominale ma di cui si può (forse) disporre solo in futuro.
L'espressione quantitativa di questa legge è data dalla seguente equazione, detta \emph{formula del valore attuale} \VA. 
\begin{equation}
  \label{eq:VA}
  \VA=\frac{C}{1+r}
\end{equation}
dove
\begin{itemize}
\item  $C$ è un flusso di cassa futuro,
\item \VA\ è il valore attuale del flusso di cassa $C$,
\item $r$ è un numero reale positivo, chiamato \emph{interesse} o \emph{rendimento} o \emph{tasso di sconto} o
  \emph{tasso di attualizzazione} o \emph{costo del capitale} o \emph{costo opportunità del capitale}, che esprime
  quantitativamente quanto è inferiore il valore di un capitale disponibile solo nel futuro rispetto al valore dello
  stesso capitale ma con disponibilità immediata. Come per i termini della definizione~\ref{def:cap} e della
  definizione~\eqref{def:inv}, i nomi di $r$ verranno da ora in avanti usati come sinonimi.
\end{itemize}

\begin{oss}\label{oss:va}
  Il valore attuale è il valore dell'investimento al tempo presente. Altri locuzioni usate per identificare il valore attuale
  sono \emph{valore iniziale}  o \emph{investimento iniziale}, \emph{capitale investito} o anche \emph{costo dell'investimento}.
\end{oss}

\begin{oss}
  Calcolare il valore attuale \VA\ di un flusso di cassa futuro si dice anche ``\emph{attualizzare il flusso di cassa}''.
\end{oss}
\begin{oss}
  Nell'applicazione della formula~\eqref{eq:VA} è implicito il fatto che la valutazione di \VA\ si fa per un capitale
  disponibile solo dopo un tempo $t$: questo tempo prende il nome di \emph{intervallo (o periodo o tempo) di capitalizzazione
    dell'investimento}. In pratica questo tempo $t$ può avere i valori più disparati od essere, in un senso che preciseremo
  nel capitolo seguente, infinitesimo.
\end{oss}
\begin{oss}
  La formula \eqref{eq:VA} viene di solito scritta nella forma più comune 
  \begin{equation}
    \label{eq:VA_prop}
    C=(1+r)\VA,\tag{\ref*{eq:VA} bis}
  \end{equation}
  che enfatizza la grandezza relativa del flusso di cassa futuro rispetto al valore attuale: la formula è presentata e analizzata
  in questa forma nella bella trattazione~\cite{Levi1957} (si noti la forma estremamente generale, chiara e concisa che essa assume
  in~\cite[§1, p. 151]{Levi1957}), dove sono discussi anche i temi del capitolo~\ref{chap:Formule_fondamentali}.
  Questa forma è però, in un certo senso, fuorviante perché fa sembrare che entrambi i capitali coinvolti siano certi e disponibili.
  In realtà $C$ è disponibile solo nel futuro ed è quindi incerto, mentre \VA\ è certo perché è quello che occorre per pagare
  l'investimento.
\end{oss}

\begin{oss}
  Le equazioni~\eqref{eq:VA} e~\eqref{eq:VA_prop} esprimono implicitamente una \emph{ipotesi di proporzionalità} tra il flusso di cassa
  futuro e il valore attuale \VA: questa ipotesi non è vera in generale (si veda ad esempio~\cite[§9, p. 158]{Levi1957}), e per rendersi
  conto di ciò basta pensare a come cambiano le possibilità di investimento disponibili in funzione dell'entità del capitale che è possibile
  investire. Tuttavia essa è sicuramente verificata all'interno di una stessa classe di investimenti: infatti in un mercato di capitali che
  sia in un certo senso ``efficiente'' (in pratica dove è possibile investire e disinvestire) nessuno accetta (se non è costretto
  ...) un investimento dove per ottenere $10$ occorre sborsare $10^2=100$ e, viceversa, nessuno propone un investimento che, a fronte
  di un gettito iniziale di $10$ lo obbligherebbe ad un esborso di $10^2=100$ allo scadere del tempo di capitalizzazione (a meno che
  il proponente non abbia intenzione di dileguarsi prima...).
\end{oss}

\begin{defn}\label{def:int}
  In accordo con la definizione~\ref{def:inv} e con l'osservazione~\ref{oss:va} \emph{assumiamo che il valore attuale \VA\ coincida
    con quanto pagato un dato investimento}. In questo caso capitale $r\VA$ in eccesso che risulta (teoricamente) disponibile al momento
  di capitalizzazione dell'investimento stesso viene chiamato \emph{interessi dell'investimento}.
\end{defn}

La terminologia introdotta dalla definizione~\ref{def:int} si mantiene anche quando (ed è quasi sempre il caso, come vederemo nel
capitolo~\ref{chap:Il_VAN} seguente) \emph{il costo dell'investimento non coincide con il suo valore \VA }.

La considerazione dell'incertezza sulla disponibilità effettiva futura dell'investimento ci porta al secondo principio fondamentale
dell'economia (\cite[§2.1.3 p. 14]{brealey_et_al1999})

\begin{legge}\label{legge:2}
  Un denaro sicuro vale più di un denaro rischioso.  
\end{legge}
La legge~\ref{legge:2} dice come cambia il valore del tasso di sconto $r$ in funzione del rischio assunto per un dato investimento:
dati due investimenti con flusso di cassa futuro $C$ dello stesso valore nominale, dei due quello più rischioso ha un valore attuale minore.
\begin{oss}
  Più chiaramente, chamati $\VA_1$ e $\VA_2$ i valori attuali di due investimenti A e B con stesso flusso di cassa futuro $C$,
  la condizione ``\emph{l'investimento A è più rischioso dell'investimento B}'' corrisponde alla condizione $\VA_1< \VA_2$,
  vale a dire
  \[
    \frac{C}{1+r_1} < \frac{C}{1+r_2} \iff r_2 < r_1.
  \]
  Quindi a parità di flusso di cassa, \emph{ad un investimento rischioso corrisponde un rendimento maggiore}.
\end{oss}

\begin{oss}
  La legge~\ref{legge:2} dice che \emph{il tasso di sconto è una misura del rischio di investimento}. In effetti, elaborando le osservazioni
  di~\cite[§7, p. 158]{Levi1957} potremmo dire che questa quantità definisce il valore attuale \VA, quindi \emph{il tasso di sconto si può
    cosiderare come previsione sui risultati degli investimenti che potrà compiere un dato investitore} (o eventualmente dei prestiti
  che potrà contrarre): diremo qualcosa di più su questo argomento nel contesto del capitolo~\ref{chap:Formule_fondamentali}, .
\end{oss}

Avendo stabilito come attualizzare flussi di cassa futuri e introdotto il concetto rischio, possiamo introdurre
la prossima legge dell'economia, che ci dice quale è la relazione tra i valori attuali di flussi di cassa
disponibili a tempi sempre più distanti nel futuro (\cite[§3.1.2 pp. 35--36]{brealey_et_al1999}).

\begin{legge}\label{legge:3}
  Un denaro domani vale più di un denaro dopodomani.
\end{legge}
La formulazione generale della legge~\ref{legge:3} non ha forma di equazione come per~\ref{legge:1}, ma di disuguaglianza:
se $\VA_{t_1}$ e $\VA_{t_2}$ sono rispettivamente i valori attuali di uno stesso flusso di
cassa $C$ disponibile a due diversi tempi di capitalizzazione $t_1$ e $t_2$ con $t_1<t_2$, deve essere necessariamente
\begin{equation}
  \label{eq:legge3}
  \VA_{t_1}>\VA_{t_2}
\end{equation}
La ragione di questa differenza è dovuta al fatto che mentre la legge~\ref{legge:1} definisce la forma matematica del
valore attuale \VA\ per un implicito (ma fissato) periodo di capitalizzazione $t$, la legge~\ref{legge:3}
stabilisce quale proprietà devono necessariamente avere tutte le generalizzazioni (ne vedremo alcune prossimo capitolo)
della prima legge a tempi $t$ arbitrari ed espliciti. Nel gergo tecnico della fisica dei mezzi continui si dice che
questa è una \emph{legge costitutiva}.
\begin{oss}
  La legge~\ref{legge:3} formalizza il fatto che non è possibile creare capitale dal nulla: l'argomento è approfondito
  nel testo \cite[§3.1.2 p. 35]{brealey_et_al1999} che illustra anche con un esempio spiritoso perché non esitono
  ``macchine  per fare soldi''.
\end{oss}
\begin{oss}
  Le leggi~\ref{legge:1},\ref{legge:2} e \ref{legge:3} possono essere sintetizzate dicendo che il valore attuale \VA\
  di un investmento è funzione strettamente monotona decrescente del suo rischio e della sua durata.
\end{oss}
