
\chapter{Leggi fondamentali}
\label{chap:Leggi_fondamentali}

\begin{legge}\label{law:0}
  Esiste un mercato dei capitali.
\end{legge}
I capitali sono risorse economiche, ``flussi di cassa'', qualcosa che per tutti ha un ``valore''
e che quindi si può comprare e scambiare. Chiameremo \emph{euro} la generica unità di misura del
capitale, valore o che dir si voglia, senza riferimento alcuno alla valuta con lo stesso nome:
avremmo potuto usare ``soldo''. Chiameremo infine \emph{investimento} l'acquisto di capitale
disponibile solo ad un tempo futuro rispetto a quello in cui si conclude l'acquisto.

\begin{legge}\label{legge:1}
  Un euro oggi vale più di un euro domani.
\end{legge}
La legge \ref{legge:1} è espressa dalla seguente formula del \emph{valore attuale}
\VA: il valore attuale di un possible un flusso di cassa futuro $C$, disponibile
al tempo futuro $t$, è
\begin{equation}
  \label{eq:VA}
  \VA=\frac{C}{1+r}
\end{equation}
dove il numero positivo $r$ è chiamato \emph{rendimento} o \emph{tasso di sconto} o
\emph{costo del capitale}. Calcolare il valore attuale \VA\ di un flusso di cassa futuro
si dice anche ``attualizzare'' il flusso di cassa.
La formula \eqref{eq:VA} viene di solito scritta nella più comune forma 
\[
  C=(1+r)\VA,
\]
che enfatizza la grandezza relativa del flusso di cassa futuro rispetto al valore attuale.
Questa forma è però, in un certo senso, fuorviante perché $C$ è incerto siccome disponibile
solo nel futuro, mentre \VA\ è certo perché disponibile ora, nel presente.

La considerazione dell'incertezza ci porta al secondo
principio dell'economia (\cite{brealey_et_al1999}, §2.1.3 p. 14)

\begin{legge}\label{legge:2}
  Un euro sicuro vale più di un euro rischioso.  
\end{legge}
La legge \ref{legge:2} dice come cambia il valore del tasso di sconto $r$ in funzione del
rischio assunto per un dato investimento: un flusso di cassa futuro, a parità di valore nominale,
ha un valore attuale maggiore se è meno rischioso quindi, all'aumentare del rischio, il valore attuale
diminuisce quindi il tasso di sconto è maggiore.

Avendo capito come attualizzare flussi di cassa futuri e introdotto il concetto rischio, possiamo introdurre
la prossima legge dell'economia, che ci dice quale è la relazione tra i rischi di flussi di cassa disponibili
a tempi sempre più distanti nel futuro.

\begin{legge}\label{legge:3}
  Un euro domani vale più un euro dopodomani.
\end{legge}
La legge \ref{legge:3} non è espressa da una formula come \eqref{eq:VA}, ma da una disuguaglianza:
se $\VA_{t_1}$ e $\VA_{t_2}$ sono rispettivamente i valori attuali di uno stesso flusso di cassa $C$
disponibile a due diversi tempi futuri $t_1$ e $t_2$ con $t_1<t_2$, allora deve essere
\begin{equation}
  \label{eq:legge3}
  \VA_{t_1}>\VA_{t_2}
\end{equation}
La legge \ref{legge:3} formalizza il fatto che non è possibile creare capitale dal nulla (\cite{brealey_et_al1999},
§3.1.2 p. 35).