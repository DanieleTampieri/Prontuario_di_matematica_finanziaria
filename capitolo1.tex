
\chapter{Leggi fondamentali}
\label{chap:Leggi_fondamentali}

\begin{legge}\label{legge:1}
  Un euro oggi vale più di un euro domani.
\end{legge}

La legge \ref{legge:1} è espressa dalla seguente formula del \emph{valore attuale}
\VA\ per un flusso di cassa  $C_1$ disponibile al tempo futuro $t_1$
\begin{equation}
  \label{eq:VA}
  \VA=\frac{C_1}{1+r_1}
\end{equation}
dove il numero (si spera) positivo $r_1$ è chiamato \emph{rendimento} o \emph{tasso di sconto} o
\emph{costo del capitale}.
La formula \eqref{eq:VA} viene comunemente messa nella forma
\[
  C_1=(1+r_1)\VA
\]
perché così si enfatizza la grandezza relativa del flusso di cassa futuro rispetto al valore attuale.
Questa forma è in un certo senso fuorviante perché mentre \VA\ è certo, è un qualcosa che esiste ora,
nel presente, $C_1$ è incerto come il futuro. E ĺa considerazione delĺincertezza ci porta al secondo
principio dell'economia

\begin{legge}\label{legge:2}
  Un euro sicuro vale più di un euro rischioso.  
\end{legge}

La legge \ref{legge:2} stabilisce come cambia il valore del tasso di sconto $r_1$ in funzione del
rischio che si assume per un dato investimento: un flusso di cassa futuro, a parità di entità nominale,
ha un valore attuale maggiore se è meno rischioso, quindi alĺaumentare del rischio il valore attuale
diminuisce quindi il tasso di sconto è maggiore.

\begin{legge}\label{legge:3}
  Un euro domani vale più un euro dopodomani.
\end{legge}

La legge \ref{legge:3} non è di solito espressa da una formula come \eqref{eq:VA} ma da una disuguaglianza:
se $\VA_{t_1}$ e $\VA_{t_2}$ sono rispettivamente i valori attuali di uno stesso flusso di cassa $C_1$ disponibile
a due tempi futuri $t_1<t_2$, allora deve essere
\begin{equation}
  \label{eq:legge2}
  \VA_{t_1}>\VA_{t_2}
\end{equation}

La legge \ref{legge:3} formalizza il fatto che non è possibile creare denaro dal nulla (\cite{brealey_et_al1999},
§3.1.2 p. 35).