%********************************************************************
%
%    Notes on the theorem of Cramer \& Wald, Newman and Besicovitc
% 
%********************************************************************
\documentclass[a4paper,10pt]{memoir}
%********************************************************************
%            Packages
%********************************************************************
\usepackage[italian]{babel}
\usepackage[utf8]{inputenc}
\usepackage[intlimits]{amsmath}
\usepackage{amsfonts}
\usepackage{amssymb}
\usepackage{mathrsfs}
\usepackage{fouriernc}
% \usepackage{mathtime}
\usepackage{IEEEtrantools}
\usepackage{cite}
\newcommand{\dm}{\mathrm{d}}
\newcommand{\Bsy}[1]{\ensuremath{{\boldsymbol{#1}}}}
\newcommand{\Prova}{\textbf{Dimostrazione}. }
\newcommand{\RNum}[1]{\MakeUppercase{\romannumeral #1}}
\newcommand{\qed}{\hfill$\blacksquare$\newline}
\newtheorem{cor}{Corollario}
\newtheorem{defn}{Definizione}
\newtheorem{legge}{Legge}
\newtheorem{lemma}{Lemma}
%\newtheorem{stackrel}
\newtheorem{thm}{Teorema}
\usepackage{hyperref}
%********************************************************************
%            Content
%********************************************************************
\author{Daniele Tampieri}
\title{Prontuario di matematica finanziaria}
\begin{document}
\maketitle

\chapter{Leggi fondamentali}
\label{chap:Leggi_fondamentali}

\begin{legge}\label{legge:1}
  Un euro oggi vale più di un euro domani.
\end{legge}

La legge \ref{legge:1} è espressa dalla seguente formula del \emph{valore attuale}
\VA\ per un flusso di cassa  $C_1$ disponibile al tempo futuro $t_1$
\begin{equation}
  \label{eq:VA}
  \VA=\frac{C_1}{1+r_1}
\end{equation}
dove il numero (si spera) positivo $r_1$ è chiamato \emph{rendimento} o \emph{tasso di sconto} o
\emph{costo del capitale}.
La formula \eqref{eq:VA} viene comunemente messa nella forma
\[
  C_1=(1+r_1)\VA
\]
perché così si enfatizza la grandezza relativa del flusso di cassa futuro rispetto al valore attuale.
Questa forma è in un certo senso fuorviante perché mentre \VA\ è certo, è un qualcosa che esiste ora,
nel presente, $C_1$ è incerto come il futuro. E ĺa considerazione delĺincertezza ci porta al secondo
principio dell'economia

\begin{legge}\label{legge:2}
  Un euro sicuro vale più di un euro rischioso.  
\end{legge}

La legge \ref{legge:2} stabilisce come cambia il valore del tasso di sconto $r_1$ in funzione del
rischio che si assume per un dato investimento: un flusso di cassa futuro, a parità di entità nominale,
ha un valore attuale maggiore se è meno rischioso, quindi alĺaumentare del rischio il valore attuale
diminuisce quindi il tasso di sconto è maggiore.

\begin{legge}\label{legge:3}
  Un euro domani vale più un euro dopodomani.
\end{legge}

La legge \ref{legge:3} non è di solito espressa da una formula come \eqref{eq:VA} ma da una disuguaglianza:
se $\VA_{t_1}$ e $\VA_{t_2}$ sono rispettivamente i valori attuali di uno stesso flusso di cassa $C_1$ disponibile
a due tempi futuri $t_1<t_2$, allora deve essere
\begin{equation}
  \label{eq:legge2}
  \VA_{t_1}>\VA_{t_2}
\end{equation}

La legge \ref{legge:3} formalizza il fatto che non è possibile creare denaro dal nulla (\cite{brealey_et_al1999},
§3.1.2 p. 35).
%\chapter{Formule fondamentali}
\label{chap:Formule_fondamentali}

Il valore attuale \VA\ di un investimento che produce un unico flusso di cassa
$C$ al tempo futuro $t$ è definito dalla formula \eqref{eq:VA}. Ad ogni investimento
è però richiesto producono flussi di cassa ad intervalli di tempo regolari, ad esempio annuali, stabiliti
: come si deve calcolare
il tasso di interesse di un flusso di cassa che verrà prodotto solo ad $n$ anni di distanza
dall'investimento? di solito un investimento produce più
flussi di cassa che si rendono disponibili a più tempi futuri.
La situazione che si crea in questi  è equivalente 
In questo caso, occorre accordarsi a priori su come definire il valore attuale
di un investimento: due scelte comuni sono le seguenti.In generale però ogni i
\begin{defn}. Un investimento ha \emph{interesse semplice} se il flusso di cassa
  che produce è sempre lo stesso.
\end{defn}
Se abbiamo un investimento che produce $n$ flussi di cassa di valore 
\begin{equation}
  \label{eq:i_semplice}
  \VA=\frac{C}{1+\sum_{i=1}^nr_i}
\end{equation}
\begin{defn}
  Un investimento 
\end{defn}


%\chapter{Il valore attuale netto e sua applicazione alle decisioni di investimento}
\chaptermark{Il valore attuale netto e sua applicazione}
\label{chap:VAN}

\section{Il valore attuale netto}
\label{sec:VAN}
\begin{equation}
  \label{eq:VAN}
  \VAN = C_0+\VA
\end{equation}



\section[L'inflazione]{Un'altra ragione per cui un denaro oggi vale più di un denaro domani: l'inflazione}
\label{sec:Infl}




%\include{capitolo4}
%\backmatter
%\include{biblio}
%\printindex
\end{document}
