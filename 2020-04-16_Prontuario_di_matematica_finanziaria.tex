%********************************************************************
%
%   Prontuario di matematica finanziaria
% 
%********************************************************************
\documentclass[a4paper,10pt]{memoir}
%********************************************************************
%            Packages
%********************************************************************
\usepackage[italian]{babel}
\usepackage[utf8]{inputenc}
\usepackage[intlimits]{amsmath}
\usepackage{amsfonts}
\usepackage{amssymb}
\usepackage{mathrsfs}
\usepackage{fouriernc}
% \usepackage{mathtime}
\usepackage{IEEEtrantools}
\usepackage{cite}
\newcommand{\dm}{\mathrm{d}}
\newcommand{\Bsy}[1]{\ensuremath{{\boldsymbol{#1}}}}
\newcommand{\Prova}{\textbf{Dimostrazione}. }
\newcommand{\RNum}[1]{\MakeUppercase{\romannumeral #1}}
\newcommand{\qed}{\hfill$\blacksquare$\newline}
\newcommand{\VA}{\ensuremath{\mathrm{VA}}}    % Valore attuale
\newcommand{\VAN}{\ensuremath{\mathrm{VAN}}}  % Valore attuale netto
\newtheorem{cor}{Corollario}
\newtheorem{defn}{Definizione}
\newtheorem{legge}{Legge}
\newtheorem{lemma}{Lemma}
%\newtheorem{stackrel}
\newtheorem{thm}{Teorema}
\usepackage{hyperref}
%********************************************************************
%            Content
%********************************************************************
\author{Daniele Tampieri}
\title{Prontuario di matematica finanziaria}
\begin{document}
\maketitle

\chapter{Leggi fondamentali}
\label{chap:Leggi_fondamentali}

\begin{legge}
  Un euro oggi vale più di un euro domani.
\end{legge}

\begin{legge}
  Un euro domani vale più un euro dopodomani.
\end{legge}
%\chapter[Il calcolo del valore attuale]{Il calcolo del valore attuale per investimenti con tempi di capitalizzazione  arbitrari}
\label{chap:Formule_fondamentali}

Il valore attuale \VA\ di un investimento che produce un flusso di cassa $C$ è stato definito nel capitolo
\ref{chap:Leggi_fondamentali} dalla formula~\eqref{eq:VA}. In questa formula è implicita la dipendenza del
valore attuale dal periodo di capitalizzazione $t$: scopo di questo capitolo è esplicitare questa dipendenza,
vale a dire mostrare come cambia il valore attuale di un investimento che genera più flussi di cassa a più
tempi futuri.
Una indicazione di quello che deve essere l'andamento qualitativo del valore attuale in funzione del tempo
di capitalizzazione è già data dalla legge \ref{legge:3}: più il flusso di cassa (ritorno di investimento)
viene pagato a tempi lontani nel futuro, più il valore attuale diminuisce come effetto dell'aumento del
rischio di investimento.
 
L'analisi che svilupperemo permetterà di valutare (attualizzare) un investimento che genera (o forse
sarebbe meglio dire ``che si attende generi'') più flussi di cassa a periodi di capitalizzazione arbitrari.

\begin{defn}\label{def:tval}
  Chiamiamo genericamente \emph{anno} il periodo unitario di capitalizzazione di un investimento.
\end{defn}

Precisiamo che il termine ``anno'' nella defninizione~\ref{def:tval} è solo convenzionale, senza nessuna
relazione con l'omonima unità di misura del tempo: il periodo di capitalizzazione considerato può in effetti
avere una durata effettiva qualsiasi, ad esempio 6 mesi, 4 settimane, un giorno, un minuto, ecc. senza che
cambi la logica dell'analisi. In effetti, come vedremo, il periodo in cui l'investimento genera un ritorno
può anche essere, in un preciso senso, infinitesimo.

In realtà più che esplicitare una dipendenza dal tempo, definiremo due possibili andamenti temporali
del valore attuale in funzione del \emph{tipo di interesse} associato a un dato investimento, vale a dire
alla struttura degli interessi (definizione~\ref{def:int}) che produce. Allo scopo notiamo che \emph{gli 
interessi prodotti da un dato investimento dipendono sempre dal costo dello stesso} (identificabile
col \VA) e dal tasso di sconto relativo al periodo di capitalizzazione, e diamo le seguenti due definizioni.

\begin{defn}\label{def:isemplice}
  Un investimento ha \emph{interesse semplice} se gli interessi che produce dipendono \emph{solamente}
  dal capitale iniziale investito e dal tasso di sconto relativo al periodo di capitalizzazione.
\end{defn}

In parole povere, la definizione~\ref{def:isemplice} dice che gli interessi prodotti dall'investimento ai tempi
precedenti non vengono usati, ossia non vengono \emph{reinvestiti} per contribuire ai flussi di cassa futuri.
Questa considerazione ci porta alla seconda definizione di tipo di interesse associato a un investimento
che consideriamo in questa sede.

\begin{defn} \label{def:icomposto}
  Un investimento ha \emph{interesse composito} (o \emph{composto}) se gli interessi che produce dipendono
  dal capitale iniziale investito, dal tasso di sconto relativo al periodo di capitalizzazione e dagli interessi
  maturati nei periodi di capitalizzazione precedenti. 
\end{defn}

In parole povere, la definizione~\ref{def:icomposto} dice che in un investimento a interesse composito gli interessi
maturati durante ogni periodo di capitalizzazione \emph{sono tacitamente reinvestiti} per contribuire ai flussi di cassa
futuri e aumentare quindi il suo valore attuale.

\section[Capitalizzazione discreta]{Il valore attuale per investimenti a capitalizzazione discreta}
\label{sec:tdiscreti}

In questa sezione mostriamo come calcolare il valore attuale di un flusso di cassa che viene generato da un investimento 
dopo un periodo di capitalizzazione complessivo pari a $n$ periodi di capitalizzazione unitari, con $n\ge 2$.

\subsection{Investimento a interesse semplice}
\label{sec:isemplice_td}

Se abbiamo un investimento con interesse semplice che produce un flusso di cassa $C_n$ dopo $n$ periodi di capitalizzazione
e rendimenti $r_i$, $i= 1, \ldots, n$ variabili per ogni periodo di capitalizzazione, il suo valore attuale \VA\ è dato dalla
seguente formula

\begin{equation}
  \label{eq:isemplice_td}
  \VA=\frac{C_n}{1+\sum_{i=1}^nr_i}.
\end{equation}
Nell'ipotesi (alquanto improbabile) che il tasso di rendimento sia costante per tutti gli $n$ periodi di capitalizzazione,
la formula \eqref{eq:isemplice_td} assume la seguente semplice forma
\[
  \VA=\frac{C_n}{1+nr}.
\]

\subsection{Investimento a interesse composito}
\label{sec:icomposto_td}

Se abbiamo un investimento con interesse composto che produce un flusso di cassa $C_n$ dopo $n$ periodi di capitalizzazione
e rendimenti $r_i$, $i= 1, \ldots, n$ variabili per ogni periodo di capitalizzazione, il suo valore attuale \VA\ è dato dalla
seguente formula
\begin{equation}
  \label{eq:icomposto_td}
  \VA=\frac{C_n}{\prod_{i=1}^n(1+r_i)}.
\end{equation}
Nell'ipotesi  (alquanto improbabile) che il tasso di rendimento sia costante per tutti gli $n$ periodi di capitalizzazione,
la formula \eqref{eq:icomposto_td} assume la seguente semplice forma
\[
  \VA=\frac{C_n}{(1+r)^n}.
\]

\begin{oss}\label{oss:isemplice}
  L'interesse semplice ha più un interesse storico che pratico, anche se a livello di certe offerte
  commerciali è ancora usato: per una breve discussione sul suo uso si può consultare questa domanda
  e risposta~\cite{Romanchuk2021} sull'Economics.StackExchange.
\end{oss}

\begin{oss}
  Quando il periodo di capitalizzazione si riduce a quello unitario, le formule~\eqref{eq:isemplice_td} e~\eqref{eq:icomposto_td}
  si riducono alla formula del valore attuale~\eqref{eq:VA}: inoltre, in entrambe i casi, confrontando i valori attuali
  calcolati su due periodi di capitalizzazione rispettivamente pari a $n$ e $m$ volte, con $n>m$, il periodo unitario, si ha
  \[
    \left.\VA\right|_{n}> \left.\VA\right|_{m}
  \]
  quindi la legge~\ref{legge:3} è verificata da entrambe queste definizioni. Esse allora rappresentano due accettabili
  dipendenze temporali del valore attuale in accordo con le leggi del capitolo~\ref{chap:Leggi_fondamentali}. 
\end{oss}

\section[Capitalizzazione continua]{Il valore attuale per investimenti a capitalizzazione continua}
\label{sec:tcontinui}
In questa sezione descriviamo come estendere le formule del capitolo precedente al caso in cui i periodi di capitalizzazione
aumentino di numero e abbiano durata temporale sempre minore. In un certo senso suddividiamo l'intervallo di capitalizzazione
$t$ in $n$ intervalli uguali in modo che la loro durata totale sia costante, vale a dire $ t=n\cdot\frac{t}{n}$, poi usiamo
questi intervalli nelle formule~\eqref{eq:isemplice_td} e~\eqref{eq:icomposto_td}, e vediamo come si trasformano quando aumentano
indefinitamente di numero. Naturalmente non svilupperemo tutti i dettagli analitici, in particolare assumendo familiare il concetto
di \emph{limite}: consigliamo ai lettori interessati ad approfondire gli aspetti puramente matematici della questione l'ottimo
(e impegnativo) libro di testo~\cite{Fischer1983}.\newline
La capitalizzazione nel continuo non costituisce, come si potrebbe pensare, una generalizzazione fine a se stessa: è invece
un modo, perfettamente legale, per sfruttare al meglio i limiti di legge sul valore massimo del rendimento dei depositi bancari.
Si può infatti dimostrare (si veda per esempio~\cite[§3.3.1, p. 45, esempio 1]{brealey_et_al1999}) che un investimento con rendimento
a tasso composto continuo del 10\% equivale a un rendimento a tasso composto su base annua del 10,5\%: siccome la legislazione
non impone un particolare periodo di capitalizzazione, la banca che offre la capitalizzazione dei depositi nel continuo offre di fatto
rendimenti leggermente più alti pur agendo in modo perfettamente legale.


\subsection{Investimento a interesse semplice}
\label{sec:isemplice_tc}

Scriviamo la formula~\eqref{eq:isemplice_td} per $n$ periodi di capitalizzazione di durata decrescente $t\over n$, 
ammettiamo tassi di sconto $r_i$, $i= 1, \ldots, n$ variabili per ogni periodo di capitalizzazione e operiamo la sostituzione
formale $r_i\mapsto {r_i\over n}$: otteniamo così la formula
\[
    \VA=\frac{C_n}{1+\sum_{i=1}^n\frac{r_i}{n}},
\]
da cui, calcolandone il limite per $n\to \infty$ grazie alla definizione di integrale di Riemann
(si veda per esempio~\cite[chapter \Romannum{13}, §\Romannum{13}.4, pp.~629--638]{Fischer1983}), segue che
\begin{equation}
  \label{eq:isemplice_tc}
  \VA=\frac{C_t}{\displaystyle{1+\int_{0}^tr(s)\dm s}}.
\end{equation}
Nell'ipotesi (alquanto improbabile) che il tasso di rendimento sia costante per tutto il periodo di capitalizzazione,
la formula \eqref{eq:isemplice_tc} si riduce praticamente alla~\eqref{eq:VA}, vale a dire
\[
  \VA=\frac{C_t}{1+rt}.
\]

\begin{oss}
  Come visto nell'esempio precedente, l'investimento con interesse semplice ha un valore attuale sostanzialmente identico
  a quello dato dalla formula~\eqref{eq:VA}: possiamo quindi affermare che~\eqref{eq:isemplice_tc} \emph{è la più semplice generalizzazione
  del valore attuale a intervalli di capitalizzazioni arbitrari}. Detto questo e memori dell'osservazione~\ref{oss:isemplice},
  non torneremo più su questo tipo di investimento.
\end{oss}

\subsection{Investimento a interesse composito}
\label{sec:icomposito_tc}

In modo analogo a quanto fatto nel paragrafo precedente, scriviamo la formula~\eqref{eq:isemplice_tc} per $n$ periodi
di capitalizzazione di durata decrescente $t\over n$ a cui corrispondono tassi di sconto $r_i$, $i= 1, \ldots, n$
variabili per ogni periodo di capitalizzazione: operando di nuovo la sostituzione formale $r_i\mapsto {r_i\over n}$,
otteniamo formula,
\[
   \VA_t=\frac{C_t}{\prod_{i=1}^n\left(1+\frac{r_i}{n}\right)}.
 \]
 Anche in questo caso passiamo al limite per $n\to\infty$ ma sia il procedimento che il risultato ottenuto sono più
 complicati:  il denominatore della formula diviene il prodotto integrale della funzione $r(t)$, introdotto da Vito
 Volterra nel 1887 (si veda~\cite{Volterra1887} oppure~\cite[pp. 291--293]{Volterra1954}). Precisamente si ha 
 \begin{equation*}
   \lim_{n\to\infty}\prod_{i=1}^n\left(1+\frac{r_i}{n}\right)=\exp\left({\displaystyle{\int_{0}^tr(s)\dm s}}\right) \triangleq \pint_{0}^{t}(1+r(s))^{\dm s}
 \end{equation*}
 (si veda ad esempio~\cite[chapter 2, §2.5, example 2.5.6, p. 35]{Slavik2007}) e di conseguenza
\begin{equation}
  \label{eq:icomposto_tc}
  \VA=\frac{C_t}{\exp\left(\int_{0}^tr(s)\dm s\right)}\triangleq\frac{C}{\displaystyle{\pint_{0}^{t}(1+r(s))^{\dm s}}}
\end{equation}
Nell'ipotesi  (alquanto improbabile) che il tasso di rendimento sia costante per tutti
gli $n$ periodi di capitalizzazione, la formula \eqref{eq:icomposto_td} assume la seguente semplice
forma:
\[
  \VA=\frac{C_t}{e^{rt}}
\]

\begin{oss}
  Con intervallo di capitalizzazione continuo e periodo di capitalizzazione unitario, ossia $t=1$, la formula~\eqref{eq:isemplice_tc}
  si riduce ancora alla formula del valore attuale~\eqref{eq:VA}, mentre questo non avviene più per la formula~\eqref{eq:icomposto_tc}:
  comunque in entrambe i casi, se si considerano due periodi di capitalizzazione $t_1<t_2$, si ha
  \[
    \left.\VA\right|_{t_2}> \left.\VA\right|_{t_1}
  \]
  quindi la legge~\ref{legge:3} è verificata da queste definizioni. Esse esplicitano perciò due possibile dipendenze temporali del valore attuale
  in accordo con le leggi del capitolo~\ref{chap:Leggi_fondamentali}. 
\end{oss}

\section{Cenni sul significato delle leggi di interesse}
\label{sec:significato}

Questa sezione è stata ispirata dalla lettura dell'articolo di Eugenio Levi~\cite{Levi1957} e presenta tre interpretazioni che gli studiosi di economia
danno delle leggi di intersse descritte nelle sezioni precedenti di questo capitolo. La prima di queste interpretazioni è una interpretazione di tipo
\emph{macroeconomico}, la seconda è quella seguita da Brian Romanchuk~\cite{Romanchuk2021}, mentre la terza ed ultima interpretazione è dovuta allo
stesso Levi.
\begin{enumerate}
\item \textbf{Interpretazione macroeconomica}: la legge di interesse esprime il valore effettivo del flusso di cassa $C$ come funzione univoca del capitale investito
  \VA, dell’istante di inzio dell’investimento e della sua durata. Si tratta quindi di una interpretazione in termini assoluti: nella concezione di chi segue questo
  tipo di interpretazione, queste leggi hanno validità simile a quella che può avere una legge fisica.
\item \textbf{Interpretazione ``commerciale''}\footnotemark{}: la legge di interesse è solo un modo per esprimere i valori del flusso di cassa $C$ e  del capitale
  investito \VA,  che sono le uniche quantità effettivamente importanti e sono definite dagli accordi tra i contraenti. In questa concezione, gli investimenti sono
  solo comuni contratti di compravendita,  dove la legge di interesse è solo uno strumento di calcolo per quei contratti dove il tempo di durata non è fissato a
  priori e può essere variato dai contraenti.
\item \textbf{Interpretazione ``probabilistica''}\footnotemark[\value{footnote}]: la legge di interesse esprime, con riferimento ad un preciso tipo di investitore,
  le previsione sui risultati degli investimenti che questo può compiere. In questa interpretazione, si può quasi dire che la legge di interesse sia una misura
  probabilistica (nel senso dato da Bruno de Finetti alla probabilità) sullo spazio degli eventi costituito dagli investimenti che un dato investitore può fare.
\end{enumerate}

Le prime due interpretazioni, molto diverse tra loro, sono in un certo modo unite nella terza: Levi, che con la sua scuola ne è il proponente, descrive in modo
chiaro e articolato come questa spiega quello che avviene nella pratica degli investimenti reali meglio delle precedenti. Naturamente non è questo il luogo per
illustrare nel dettaglio queste interpretazioni, ma è comunque utile capire che i concetti descritti in queste note non sono noti a priori ma sono frutto di abili
e profonde analisi.

\footnotetext[1]{Il termine virgolettato non è usato da Levi, anche se l'interpretazione associata è ovviamente considerata dallo studioso: ho deciso
  di usarelo come ausilio mnemonico e per comodità di linguaggio.}
%\chapter{Il valore attuale netto e sua applicazione alle decisioni di investimento}
\chaptermark{Il valore attuale netto e sua applicazione}
\label{chap:Il_VAN}


%\include{capitolo4}
%\backmatter
\bibliographystyle{osajnl}
\bibliography{2020-04-16_prontuario.bib}
% \include{biblio}
%\printindex
\end{document}
