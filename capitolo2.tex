\chapter{Formule fondamentali}
\label{chap:Formule_fondamentali}

Il valore attuale \VA\ di un investimento che produce un unico flusso di cassa
$C$ al tempo futuro $t$ è definito dalla formula \eqref{eq:VA}. Ad ogni investimento
è però richiesto producono flussi di cassa ad intervalli di tempo regolari, ad esempio annuali, stabiliti
: come si deve calcolare
il tasso di interesse di un flusso di cassa che verrà prodotto solo ad $n$ anni di distanza
dall'investimento? di solito un investimento produce più
flussi di cassa che si rendono disponibili a più tempi futuri.
La situazione che si crea in questi  è equivalente 
In questo caso, occorre accordarsi a priori su come definire il valore attuale
di un investimento: due scelte comuni sono le seguenti.In generale però ogni i
\begin{defn}. Un investimento ha \emph{interesse semplice} se il flusso di cassa
  che produce è sempre lo stesso.
\end{defn}
Se abbiamo un investimento che produce $n$ flussi di cassa di valore 
\begin{equation}
  \label{eq:i_semplice}
  \VA=\frac{C}{1+\sum_{i=1}^nr_i}
\end{equation}
\begin{defn}
  Un investimento 
\end{defn}

