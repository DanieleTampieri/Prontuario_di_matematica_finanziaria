\chapter{Formule di calcolo}
\label{chap:Formule_fondamentali}

Il valore attuale \VA\ di un investimento che produce un flusso di cassa
$C$ in un periodo di tempo $t$ è stato definito nel capitolo \ref{chap:Leggi_fondamentali}
dalla formula \eqref{eq:VA}. In detta formula la dipendenza dal tempo $t$ è però implicita:
è lo scopo di questo capitolo rendere esplicita questa dipendenza, quindi mostrare come cambia
il valore attuale di un investimento quando viene considerato per periodi più lungh del tempo
di riferimento ceonvenuto. Una indicazione qualitativa è gia data dalla la legge \ref{legge:3}:
più il ritorno di investimento viene pagato lontano nel tempo, più il valore attuale diminuisce
come effetto della legge \ref{legge:1}.
 
non è quasi. sempre considerato  necessario
poter valutare (attualizzare) un investimento che genera flussi di cassa durante
due o più periodi successivi: lo scopo di questa sezione è generalizzare la formula
del valore attuale in tal senso. Chiameremo \emph{anno} il sigolo periodo di valutazione,
senza nessun riferimento alla unità di misura del tempo che ha lo stesso nome: il periodo
di valutazione considerato può avere durata ad esempio di 6 mesi, 1 mese, 4 settimane, un
giorno, un minuto, senza che cambi la logica del discorso

Iniziamo definendo come si deve calcolare il valore attuale di un flusso di cassa
prodotto solo dopo $n$ periodi di valutazione, con $n\ge 2$:

In questo caso, occorre accordarsi a priori su come definire il valore attuale
di un investimento: due scelte comuni sono le seguenti.In generale però ogni i
\begin{defn}. Un investimento ha \emph{interesse semplice} se il flusso di cassa
  che produce è sempre lo stesso.
\end{defn}
Se abbiamo un investimento che produce $n$ flussi di cassa di valore 
\begin{equation}
  \label{eq:isemplice}
  \VA=\frac{C}{1+\sum_{i=1}^nr_i}
\end{equation}
Nell'ipotesi  (alquanto improbabile) che il tasso di rendimento sia costante per tutti
gli $n$ periodi di valutazione, la formula \eqref{eq:isemplice} assume la seguente semplice
forma:
\[
  \VA=\frac{C}{1+nr}
\]
\begin{defn}
  Un investimento ha interesse composito sen
\end{defn}
Se abbiamo un investimento a interesse semplice che produce un flusso di cassa $C$ dopo
$n$ anni
\begin{equation}
  \label{eq:icomposto}
  \VA=\frac{C}{\prod_{i=1}^n(1+r_i)}
\end{equation}
Nell'ipotesi  (alquanto improbabile) che il tasso di rendimento sia costante per tutti
gli $n$ periodi di valutazione, la formula \eqref{eq:isemplice} assume la seguente semplice
forma:
\[
  \VA=\frac{C}{(1+r)^n}
\]

