\chapter{Formule di calcolo}
\label{chap:Formule_fondamentali}

Il valore attuale \VA\ di un investimento che produce un flusso di cassa
$C$ nel periodo $t$ è definito dalla formula \eqref{eq:VA}. È però necessario
poter valutare (attualizzare) un investimento che  genera flussi di cassa durante
due o più periodi successivi: lo scopo di questa sezione è generalizzare la formula
del valore attuale in tal senso.

Iniziamo definendo come si deve calcolare il valore
attuale di un flusso di cassa che viene prodotto solo dopo $n$ periodi di valutazione,
con $n\ge 2$. 
In questo caso, occorre accordarsi a priori su come definire il valore attuale
di un investimento: due scelte comuni sono le seguenti.In generale però ogni i
\begin{defn}. Un investimento ha \emph{interesse semplice} se il flusso di cassa
  che produce è sempre lo stesso.
\end{defn}
Se abbiamo un investimento che produce $n$ flussi di cassa di valore 
\begin{equation}
  \label{eq:i_semplice}
  \VA=\frac{C}{1+\sum_{i=1}^nr_i}
\end{equation}
Nell'ipotesi  (alquanto improbabile) che il tasso di rendimento sia costante per tutti
gli $n$ periodi di valutazione, la formula \eqref{eq:i_semplice} assume la seguente semplice
forma:
\[
  \VA=\frac{C}{1+nr}
\]
\begin{defn}
  Un investimento ha interesse composito 
\end{defn}

